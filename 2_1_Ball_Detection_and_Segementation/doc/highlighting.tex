\usepackage{listings}
\usepackage{color}


\definecolor{gray}{rgb}{0.41, 0.41, 0.41}
\definecolor{purple}{rgb}{0.0, 0.0, 1.0}
\definecolor{green}{rgb}{0.0, 0.5, 0.0}

\definecolor{mygreen}{rgb}{0,0.6,0}
\definecolor{mygray}{rgb}{0.5,0.5,0.5}
\definecolor{mymauve}{rgb}{0.58,0,0.82}


% ========= minted settings =========
% for minted package - grey background color
\definecolor{codeBgColor}{rgb}{0.95, 0.95, 0.95}

% Standard-Einstellungen für minted
\setminted{
    %frame=lines, % Rahmen um das Listing
    framesep=2mm, % Abstand zwischen Rahmen und Code
    baselinestretch=1.2, % Zeilenabstand
    fontsize=\footnotesize, % Schriftgröße
    linenos=true, % Zeilennummern anzeigen
    breaklines=true, % Zeilenumbrüche zulassen
    breakautoindent=true, % Automatisches Einrücken bei Zeilenumbruch
    autogobble=true, % Einrücken von Leerzeichen automatisch
    python3=true, % Sprache auf Python 3 setzen
    bgcolor=codeBgColor % Hintergrundfarbe definieren
    % Weitere Optionen nach Bedarf
}

% ========= lst listings settings =========
\lstdefinelanguage{pseudoCode}
{
	morekeywords={and, array, begin, case, const, div, do, downto, else, end, file, for, function, goto, if, in, label, mod, nil, not, of, or, packed, procedure, program, record, repeat, set, then, to, type, until, var, while, with, write, writeln, read, funktion, procedure},
	morecomment=[l]{//},
	morestring=[b]',
	sensitive=false,
}

%\lstset{
%	columns=flexible,
%	keepspaces=true,
%	showstringspaces=false,
%	basicstyle=\normalsize\ttfamily,
%	commentstyle=\color{gray},
%	keywordstyle=\color{purple},
%	stringstyle=\color{green},
%	stepnumber=1,
%	numbers=left,
%	mathescape=true,
%	literate={\	}{{\ }}4
%}

\lstset{ 
  backgroundcolor=\color{white},   % choose the background color; you must add \usepackage{color} or \usepackage{xcolor}; should come as last argument
  basicstyle=\fontfamily{pcr}\footnotesize,        % the size of the fonts that are used for the code
  breakatwhitespace=false,         % sets if automatic breaks should only happen at whitespace
  breaklines=true,                 % sets automatic line breaking
  captionpos=b,                    % sets the caption-position to bottom
  commentstyle=\color{mygreen},    % comment style
  deletekeywords={...},            % if you want to delete keywords from the given language
  escapeinside={\%*}{*)},          % if you want to add LaTeX within your code
  extendedchars=true,              % lets you use non-ASCII characters; for 8-bits encodings only, does not work with UTF-8
  firstnumber=1,                   % start line enumeration with line 1000
  frame=single,	                   % adds a frame around the code
  keepspaces=true,                 % keeps spaces in text, useful for keeping indentation of code (possibly needs columns=flexible)
  keywordstyle=\color{blue},       % keyword style
  language=C,	                     % the language of the code
  morekeywords={*,...},            % if you want to add more keywords to the set
  numbers=left,                    % where to put the line-numbers; possible values are (none, left, right)
  numbersep=5pt,                   % how far the line-numbers are from the code
  numberstyle=\tiny\color{mygray}, % the style that is used for the line-numbers
  rulecolor=\color{black},         % if not set, the frame-color may be changed on line-breaks within not-black text (e.g. comments (green here))
  showspaces=false,                % show spaces everywhere adding particular underscores; it overrides 'showstringspaces'
  showstringspaces=false,          % underline spaces within strings only
  showtabs=false,                  % show tabs within strings adding particular underscores
  stepnumber=1,                    % the step between two line-numbers. If it's 1, each line will be numbered
  stringstyle=\color{mymauve},     % string literal style
  tabsize=2,	                     % sets default tabsize to 2 spaces
  title=\lstname                   % show the filename of files included with \lstinputlisting; also try caption instead of title
}

\lstdefinestyle{customc}{
  belowcaptionskip=1\baselineskip,
  breaklines=true,
  frame=L,
  xleftmargin=\parindent,
  language=C,
  showstringspaces=false,
  basicstyle=\footnotesize\ttfamily,
  keywordstyle=\bfseries\color{green!40!black},
  commentstyle=\itshape\color{purple!40!black},
  identifierstyle=\color{blue},
  stringstyle=\color{orange},
}

\lstset{literate=%
  {Ö}{{\"O}}1
  {Ä}{{\"A}}1
  {Ü}{{\"U}}1
  {ß}{{\ss}}1
  {ü}{{\"u}}1
  {ä}{{\"a}}1
  {ö}{{\"o}}1
}


% XML syntax highlighting
\definecolor{gray}{rgb}{0.4,0.4,0.4}
\definecolor{darkblue}{rgb}{0.0,0.0,0.6}
\definecolor{cyan}{rgb}{0.0,0.6,0.6}

% \lstset{
%   basicstyle=\ttfamily,
%   columns=fullflexible,
%   showstringspaces=false,
%   commentstyle=\color{gray}\upshape
% }

\lstdefinelanguage{XML}
{
  morestring=[b]",
  morestring=[s]{>}{<},
  morecomment=[s]{<?}{?>},
  stringstyle=\color{black},
  identifierstyle=\color{darkblue},
  keywordstyle=\color{cyan},
  morekeywords={xmlns}, % list your attributes here
  breaklines=true,
  tabsize=4
}

\lstdefinelanguage{DTD}
{
  morestring=[b]",
  morestring=[s]{>}{<},
  morecomment=[s]{<?}{?>},
  stringstyle=\color{black},
  identifierstyle=\color{darkblue},
  keywordstyle=\color{cyan},
  morekeywords={xmlns,type}
  keepspaces=true,
  tabsize=4
}

