\section{Segmentierung der Bälle}

Nach der Klassifizierung der Bälle folgt die Segmentierung. Die Segmentierung basiert hauptsächlich auf der Farbe der Bälle. Im Rahmen dieses Projekts konzentrieren werden Fußbälle segmentiert.

Es treten jedoch einige Herausforderungen auf:
\begin{itemize}
    \item Fußbälle haben nicht immer die gleiche Farbe. Sie sind meistens weiß, jedoch variiert dies.
    \item Fußbälle sind nicht einfärbig, sondern bestehen oft aus mehreren Farbflächen.
    \item Bälle werfen Schatten, besonders wenn sie auf dem Boden liegen. Diese Schatten führen dazu, dass etwa ein Drittel des Balls deutlich dunkler erscheint, was die Segmentierung erschwert.
\end{itemize}

Um eine präzise Segmentierung zu erzielen, müssen diese Herausforderungen berücksichtigt werden. Dafür werden verschiedene Methoden getestet, die unterschiedliche Ansätze zur Farbsegmentierung nutzen.


\subsection{Information zu den Tests}

In dieser Dokumentation wird für Vergleichszwecke dasselbe Bild verwendet. Es wurde jedoch auch mit mehreren anderen Bildern getestet, die im Anhang der Abgabe enthalten sind.



\subsection{Aufteilung der Implementierung}

Die Implementierung der Ballsegmentierung erfolgt in mehreren Schritten:
\begin{enumerate}
    \item Einlesen oder Erstellen der YOLO-Ergebnisse (Sportball-Bounding-Boxes und Klassifizierungsbild).
    \item Erstellung der Ball-Maske (verschiedene Methoden zur Segmentierung werden getestet).
    \item Overlay der Ball-Maske auf das Originalbild.
    \item Anzeige und Speicherung der bearbeiteten Bilder.
\end{enumerate}

\newpage
\subsection{Erstellung der Ball-Maske}

Zur Erstellung der Ball-Maske wird der vom YOLO-Modell erkannte Bereich des Balls genauer untersucht. Dieser Bereich wird um 10 Pixel in alle Richtungen erweitert, um eventuelle Fehler durch abgeschnittene Bälle zu kompensieren.

Es werden mehrere Methoden zur Segmentierung getestet:


\subsubsection{Farbsegmentierung}

Diese Methode nutzt ausschließlich die Farbe des Balls zur Segmentierung. Dabei wird ein Schwellenwert für die Farbe des Balls festgelegt, wobei in dieser Methode nur weiße Bälle segmentiert werden können. Andere Farben werden nicht berücksichtigt.

\begin{itemize}
    \item Filterung:
    \begin{enumerate}
        \item Festlegung der Farbgrenzen für den weißen Bereich des Balls.
        \item Weichzeichnung des Bildes, um Rauschen zu reduzieren.
        \item Erzeugung eines Farb-Schwellenwerts, der idealerweise nur die Ballfarbe (weiß) enthält.
        \item Anwendung eines Schwellenwerts auf das Bild, um eine binäre Schwarz-Weiß-Maske zu erstellen.
        \item Optional: Einsatz morphologischer Filter, um Artefakte und Rauschen zu entfernen.
    \end{enumerate}
\end{itemize}


\begin{minipage}[t]{0.4\textwidth}
    \includeImgNoUrl{H}{width=0.4\linewidth}{img/3_ball_maske_farbe.png}{Ballmaske mit Farbsegmentierung}{fig1}{\centering}
\end{minipage}
%
\begin{minipage}[t]{0.6\textwidth}
    \includeImgNoUrl{H}{width=0.8\linewidth}{img/3_ball_maske_farbe_1.png}{Ball Overlay mit Farbsegmentierung}{fig2}{\centering}
\end{minipage}


\newpage

\subsubsection{KMeans-Segmentierung}

Bei dieser Methode wird das Bild zunächst mit Filtern vorbereitet, bevor der KMeans-Algorithmus auf die Bilddaten angewendet wird, um die Farben zu segmentieren. Diese Methode liefert gute Ergebnisse, solange die Filter und Parameter für das jeweilige Bild passend gewählt werden. Sie ist jedoch nicht universell anwendbar, da sie stark von der Farbgebung des Balls abhängt.

\begin{itemize}
    \item Filterung:
    \begin{enumerate}
        \item Das Bild wird mit einem Gaussian-Blur verschwommen gemacht.
        \item Der KMeans-Algorithmus wird zur Segmentierung angewendet.
        \item Das Bild wird in Graustufen umgewandelt.
        \item Ein Schwellenwert wird auf das Graustufenbild angewendet, um es in ein binäres Schwarz-Weiß-Bild umzuwandeln.
        \item Morphologische Filter (Open/Close) werden eingesetzt, um Artefakte zu entfernen.
    \end{enumerate}
\end{itemize}

\begin{minipage}[t]{0.4\textwidth}
    \includeImgNoUrl{H}{width=0.4\linewidth}{img/3_ball_maske_kmeans.png}{Ballmaske mit KMeans-Segmentierung}{fig3}{\centering}
\end{minipage}
%
\begin{minipage}[t]{0.6\textwidth}
    \includeImgNoUrl{H}{width=0.8\linewidth}{img/3_ball_maske_kmeans_1.png}{Ball Overlay KMeans-Segmentierung}{fig4}{\centering}
\end{minipage}


\subsubsection{Kantenerkennung mit Canny- und Sobel-Edge-Detektoren}

Diese Methode liefert keine brauchbaren Ergebnisse und wird daher nicht weiter verfolgt. Die Kantenerkennung konnte keine zufriedenstellende Segmentierung des Balls erreichen.

\newpage
\subsubsection{Kreiserkennung mit HoughCircles}

Diese Methode liefert ausgezeichnete Ergebnisse. Die Bälle werden unabhängig von ihrer Farbe oder etwaigen Spiegelungen zuverlässig erkannt. Dies wird durch die Anwendung der Hough-Transformation zur Kreiserkennung erreicht.

\begin{itemize}
    \item Filterung:
    \begin{enumerate}
        \item Das Bild wird in Graustufen umgewandelt.
        \item Die \texttt{HoughCircles}-Funktion erkennt Bälle im Bild.
        \item Eine binäre Weiß-Schwarz-Maske wird für die erkannten Kreise erstellt.
    \end{enumerate}
\end{itemize}

\begin{minipage}[t]{0.4\textwidth}
    \includeImgNoUrl{H}{width=0.4\linewidth}{img/3_ball_maske_houghcircles.png}{Ballmaske mit HoughCircles}{fig5}{\centering}
\end{minipage}
%
\begin{minipage}[t]{0.6\textwidth}
    \includeImgNoUrl{H}{width=0.8\linewidth}{img/3_ball_maske_houghcircles_1.png}{Ball Overlay mit HoughCircles}{fig6}{\centering}
\end{minipage}

\newpage
\subsection{Ball im Bild segmentieren}

Nach der Erstellung der Ball-Maske wird diese auf das Originalbild angewendet, um den Ballbereich zu kennzeichnen. Dies geschieht durch ein Overlay der Maske mit einer roten Farbe, um den Ball visuell hervorzuheben.

\begin{minted}[linenos]{python}
image_ball_box = image[y_min:y_max, x_min:x_max].copy()
overlay_color = np.full_like(image_ball_box, (0, 0, 255), dtype=np.uint8)

mask_indices = ball_mask > 0
image_ball_box[mask_indices] = ((1 - alpha) * image_ball_box[mask_indices] + alpha * overlay_color[mask_indices]).astype(np.uint8)
\end{minted}

\begin{minipage}[t]{0.3\textwidth}
    \includeImgNoUrl{H}{width=1.0\linewidth}{img/3_ball_maske_farbe_ergebnis.png}{Ergebnis mit Farbsegmentierung}{fig2}{\centering}
\end{minipage}
%
\begin{minipage}[t]{0.3\textwidth}
    \includeImgNoUrl{H}{width=1.0\linewidth}{img/3_ball_maske_kmeans_ergebnis.png}{Ergebnis mit KMeans-Segmentierung}{fig4}{\centering}
\end{minipage}
%
\begin{minipage}[t]{0.3\textwidth}
    \includeImgNoUrl{H}{width=1.0\linewidth}{img/3_ball_maske_houghcircles_ergebnis.png}{Ergebnis mit HoughCircles}{fig6}{\centering}
\end{minipage}

\subsection{Zusammenfassung Implementierung}

Die Segmentierung der Bälle wurde erfolgreich umgesetzt, indem verschiedene Ansätze zur Farbbasierenden Segmentierung getestet und miteinander verglichen wurden. Die Kreiserkennung mittels \texttt{HoughCircles} hat sich als die zuverlässigste Methode erwiesen, um die Bälle präzise zu segmentieren, unabhängig von ihrer Farbe und Position im Bild.
