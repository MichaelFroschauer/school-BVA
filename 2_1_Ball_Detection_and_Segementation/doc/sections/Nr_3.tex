\section{Segmentierung der Bälle}

Im nächsten Schritt sollten nach der Klassifizierung der Bälle, die Segmentierung folgen. Hierfür sollte haupsächlich die Farbe des Balls hergenommen werden.

In unserem Fall sollten Fußbälle segmentiert werden.

Hier entstehen einige Schwierigkeiten:
* Fußbälle haben typischerweise nicht immer die gleiche Farbe. Meistens sind sie weiß, allerdings nicht immer.
* Fußbälle sind nicht einfärbig. Es gibt oft verschiedene Bereiche die eine andere Farbe haben.
* Bälle im allgemeinen, sind auf ca 1/3 der Fläche des Bildes um einiges dunkler, da sie einen Schatten werfen, gerade wenn diese am Boden liegen. Das stellt eine Herausforderung bei der Segmentierung mittels der Farbe dar.

Um die Bälle möglichst gut segmentieren zu können muss auf diese Schwierigkeiten geachtet werden und entsprechend verschiedene Ansätze getestet, die ein möglichst gute Segmentierung versprechen.


## Aufteilung der Implementierung

1. Erstellen oder einlesen des Ergebnisses vom YOLO Modell (sports ball boxes, image mit klassifizierung).
2. Erstellen der Ball-Maske (Hier werden verschiedene Methoden getestet)
3. Overlay der Ball-Maske auf das Originalbild erstellen.
4. Erstellte Bilder anzeigen/speichern.



## Erstellung der Ball-Maske

Für die Erstellung der Ball-Maske wurde der Bereich der von YOLO-Modell erkannt wurde näher analysiert. Dabei wurde nicht genau der Bereich gewählt sondern, der Bereich in jede Richtung um 10 Pixel erweitert. Das hat den Grund, da das YOLO-Modell manche Bälle leicht abgeschnitten hat.

Ich habe hierfür verschiedene Methoden getestet um die Bälle zu segmentieren.
* Bild entsprechend mit Filtern vorbereiten und dann mit KMeans Algorithmus nach Farben segmentieren.
    -> Wenn die Filter und Filterparameter entsprechend dem Bild gut gewählt werden funktioniert es relativ gut. Allerdings dann nicht allgemein für verschiedene Bälle mit verschiedenen Farben.
    -> Filterung:
        1. Bild verschwommen machen mit Gaussian-Blur.
        2. KMeans-Segmentierung anwenden.
        3. Bild in Graustufenbild umwandeln.
        4. Threshold auf das Bild anwenden um es Weiß-Schwarz zu machen.
        5. Morphologische Filter Open-Close anwenden um Artefakte zu entfernen.
* Kantenerkennung mit Canny-Edge-Detector und Sobel-Edge-Detector
    -> Keine guten Ergebnisse, man konnte damit garnichts anfangen. Nicht weiter verfolgt.
* Bild entsprechend mit Filtern vorbereiten und dann Kreiserkennung mit `cv2.HoughCircles` den Ball erkennen.
    -> Funktioniert ausgezeichnet gut. Die Bälle werden unabhängig von Spiegelung und Farbe sehr gut erkannt.
    -> Filterung:
        1. Bild in Graustufenbild umwandeln.
        2. Bälle im Bild mit `HoughCircles` erkennen lassen.
        3. Weiß-Schwarz Maske des erkannten Kreis erstellen.



## Ball im Bild segmentieren

Nach der Erstellung der Ball-Maske kann diese auf das ursprüngliche Bild angewandt werden. 
Hierfür wird ... (ganz kurze Beschreibung des Codes, dieser muss nicht ganz eingefügt werden):

```python
image_ball_box = image[y_min:y_max, x_min:x_max].copy()

# Create red overlay color for the same region
overlay_color = np.full_like(image_ball_box, (0, 0, 255), dtype=np.uint8)

# Apply mask to the overlay
mask_indices = ball_mask > 0
image_ball_box[mask_indices] = ((1 - alpha) * image_ball_box[mask_indices] + alpha * overlay_color[mask_indices]).astype(np.uint8)

# Place modified region back into the original image
image_with_overlay = image.copy()
image_with_overlay[y_min:y_max, x_min:x_max] = image_ball_box
```






