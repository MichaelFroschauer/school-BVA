\section{Segmentierung der Bälle}

Im nächsten Schritt sollten nach der Klassifizierung der Bälle, die Segmentierung folgen. Hierfür sollte haupsächlich die Farbe des Balls hergenommen werden.

In unserem Fall sollten Fußbälle segmentiert werden.

Hier entstehen einige Schwierigkeiten:
* Fußbälle haben typischerweise nicht immer die gleiche Farbe. Meistens sind sie weiß, allerdings nicht immer.
* Fußbälle sind nicht einfärbig. Es gibt oft verschiedene Bereiche die eine andere Farbe haben.
* Bälle im allgemeinen, sind auf ca 1/3 der Fläche des Bildes um einiges dunkler, da sie einen Schatten werfen, gerade wenn diese am Boden liegen. Das stellt eine Herausforderung bei der Segmentierung mittels der Farbe dar.

Um die Bälle möglichst gut segmentieren zu können muss auf diese Schwierigkeiten geachtet werden und entsprechend verschiedene Ansätze getestet, die ein möglichst gute Segmentierung versprechen.


## Aufteilung der Implementierung

1. Erstellen oder einlesen des Ergebnisses vom YOLO Modell (sports ball boxes, image mit klassifizierung).
2. Erstellen der Ball-Maske (Hier werden verschiedene Methoden getestet)
3. Overlay der Ball-Maske auf das Originalbild erstellen.
4. Erstellte Bilder anzeigen/speichern.



## Erstellung der Ball-Maske

Für die Erstellung der Ball-Maske wurde der Bereich der von YOLO-Modell erkannt wurde näher analysiert. Dabei wurde nicht genau der Bereich gewählt sondern, der Bereich in jede Richtung um 10 Pixel erweitert. Das hat den Grund, da das YOLO-Modell manche Bälle leicht abgeschnitten hat.

Hierfür werden verschiedene Methoden getestet um die Bälle zu segmentieren.
* Bild entsprechend mit Filtern vorbereiten und dann mit KMeans Algorithmus nach Farben segmentieren.
    -> Wenn die Filter und Filterparameter entsprechend dem Bild gut gewählt werden funktioniert es relativ gut. Allerdings dann nicht allgemein für verschiedene Bälle mit verschiedenen Farben.
    -> Filterung:
        1. Bild verschwommen machen mit Gaussian-Blur.
        2. KMeans-Segmentierung anwenden.
        3. Bild in Graustufenbild umwandeln.
        4. Threshold auf das Bild anwenden um es Weiß-Schwarz zu machen.
        5. Morphologische Filter Open-Close anwenden um Artefakte zu entfernen.

... (ganz kurze Beschreibung des Codes, dieser muss nicht ganz eingefügt werden):
```python
def create_ball_mask_with_kmeans(image, ball_box, k=3, threshold_value=100):
    """
    Creates a binary mask for the ball in the image using K-Means clustering.

    :param image (np.ndarray): The input image (BGR format).
    :param ball_box (tuple): Coordinates of the bounding box [x_min, y_min, x_max, y_max].
    :param k (int): Number of clusters for K-Means segmentation (default is 3).
    :param threshold_value (int): The threshold value to create a binary mask (default is 100).

    :return np.ndarray: The binary mask of the ball.
    """

    x_min, y_min, x_max, y_max = ball_box
    image_ball_box = image[y_min:y_max, x_min:x_max]

    # Apply Gaussian Blur for better K-Means results
    kernel_size = get_dynamic_kernel_size(image_ball_box, 0.1)
    print(kernel_size)
    blurred_image = cv2.GaussianBlur(image_ball_box, kernel_size, 0)

    cv2.imshow("blurred_image", blurred_image)

    # Apply K-Means to segment the image
    image_segmented = apply_kmeans(blurred_image, k, 50)

    cv2.imshow("image_segmented", image_segmented)

    # Convert to grayscale for thresholding
    gray_segmented = cv2.cvtColor(image_segmented, cv2.COLOR_BGR2GRAY)

    # Apply thresholding
    _, ball_mask = cv2.threshold(gray_segmented, threshold_value, 255, cv2.THRESH_BINARY)

    kernel = cv2.getStructuringElement(cv2.MORPH_ELLIPSE, (5, 5))
    ball_mask = cv2.morphologyEx(ball_mask, cv2.MORPH_OPEN, kernel)   # Removes noise
    ball_mask = cv2.morphologyEx(ball_mask, cv2.MORPH_CLOSE, kernel)  # Closes gaps

    cv2.imshow("ball_mask", ball_mask)
    return ball_mask
```

* Kantenerkennung mit Canny-Edge-Detector und Sobel-Edge-Detector
    -> Keine guten Ergebnisse, man konnte damit garnichts anfangen. Nicht weiter verfolgt.
* Bild entsprechend mit Filtern vorbereiten und dann Kreiserkennung mit `cv2.HoughCircles` den Ball erkennen.
    -> Funktioniert ausgezeichnet gut. Die Bälle werden unabhängig von Spiegelung und Farbe sehr gut erkannt.
    -> Filterung:
        1. Bild in Graustufenbild umwandeln.
        2. Bälle im Bild mit `HoughCircles` erkennen lassen.
        3. Weiß-Schwarz Maske des erkannten Kreis erstellen.

... (ganz kurze Beschreibung des Codes, dieser muss nicht ganz eingefügt werden):
```python
def create_ball_mask_with_edge_detection(image, ball_box):
    """
    Detects circles within a specified bounding box in an image using edge detection and Hough Circle Transform.

    :param image (numpy.ndarray): The input image (BGR format).
    :param ball_box (tuple): A tuple (x_min, y_min, x_max, y_max) representing the coordinates of the bounding box.

    :return tuple:
            - ball_mask (numpy.ndarray): A binary mask of the detected circles within the bounding box.
            - detected_circles (list): A list of tuples, where each tuple represents a detected circle
              in the format (x, y, radius).
    """
    # Bounding box coordinates
    x_min, y_min, x_max, y_max = ball_box

    # Crop the region of interest (ROI) from the original image based on the bounding box
    image_ball_box = image[y_min:y_max, x_min:x_max].copy()

    # Convert the cropped image to grayscale for edge detection
    gray = cv2.cvtColor(image_ball_box.copy(), cv2.COLOR_BGR2GRAY)

    # Apply Gaussian blur to reduce noise and improve circle detection
    gray_blurred = cv2.GaussianBlur(gray, (15, 15), 0)

    # Detect circles using Hough Circle Transform
    # - dp: Inverse ratio of resolution
    # - minDist: Minimum distance between detected centers
    # - param1: Upper threshold for Canny edge detector
    # - param2: Threshold for center detection
    # - minRadius and maxRadius: Limits for circle size
    circles = cv2.HoughCircles(
        gray_blurred, cv2.HOUGH_GRADIENT, dp=1.4, minDist=100,
        param1=50, param2=30, minRadius=10, maxRadius=500
    )

    ball_mask = np.zeros_like(image_ball_box)
    detected_circles = []
    if circles is not None:
        circles = np.round(circles[0, :]).astype("int")
        for (x, y, r) in circles:
            # Draw a filled white circle on the mask at the detected location
            cv2.circle(ball_mask, (x, y), r, (255, 255, 255), -1)
            detected_circles.append((x, y, r))

    return ball_mask, detected_circles
```



## Ball im Bild segmentieren

Nach der Erstellung der Ball-Maske kann diese auf das ursprüngliche Bild angewandt werden. 
Hierfür wird ... (ganz kurze Beschreibung des Codes, dieser muss nicht ganz eingefügt werden):

```python
image_ball_box = image[y_min:y_max, x_min:x_max].copy()

# Create red overlay color for the same region
overlay_color = np.full_like(image_ball_box, (0, 0, 255), dtype=np.uint8)

# Apply mask to the overlay
mask_indices = ball_mask > 0
image_ball_box[mask_indices] = ((1 - alpha) * image_ball_box[mask_indices] + alpha * overlay_color[mask_indices]).astype(np.uint8)

# Place modified region back into the original image
image_with_overlay = image.copy()
image_with_overlay[y_min:y_max, x_min:x_max] = image_ball_box
```






