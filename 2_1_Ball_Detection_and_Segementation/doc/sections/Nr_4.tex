\section{Statistische Analyse}

In dieser Analyse werden die YOLO-basierten Bounding Boxes und die durch Segmentierung ermittelten Ball-Zentren verglichen. Für beide Methoden werden die Schwerpunkte berechnet und die Abweichung zwischen den Ergebnissen statistisch ausgewertet.

\subsection{Berechnung des Schwerpunkts}
Für die YOLO-Bounding-Box wird der Schwerpunkt als Zentrum der Box berechnet:
\[
x_{\text{YOLO}} = \frac{\text{width}_{\text{box}}}{2}, \quad y_{\text{YOLO}} = \frac{\text{height}_{\text{box}}}{2}
\]
Die Hough-Transformation liefert bereits den Schwerpunkt des erkannten Kreises direkt, sodass hier keine weitere Berechnung erforderlich ist.

\subsection{Statistische Auswertung}
Die Euclidische Distanz zwischen den Schwerpunkten der YOLO-Bounding-Box und den Hough-Zentren wird berechnet und für jede Testbildpaarung gespeichert. Die statistische Analyse umfasst die Berechnung des Mittelwerts, der Standardabweichung und des Medians der Distanzen:

\begin{itemize}
    \item Mittelwert: 1.9946 Pixel
    \item Standardabweichung: 1.0312 Pixel
    \item Median: 1.4977 Pixel
\end{itemize}

Die statistische Auswertung zeigt eine mittlere Abweichung von etwa 2 Pixeln zwischen den Schwerpunkten der YOLO Bounding Boxen und den Schwerpunkten aus der Hough-Transformation.

Die Ergebnisse der Analyse werden durch Bilder veranschaulicht, auf denen die Schwerpunkte eingezeichnet sind. Der rote Punkt stellt den YOLO-Schwerpunkt dar, der grüne Punkt den Schwerpunkt aus der Hough-Transformation.

\begin{table}[H]
    \centering
    \begin{tabular}{|c|c|c|c|c|}
    \hline
    \textbf{Ball} & \textbf{YOLO Centroid} & \textbf{Hough Centroid} & \textbf{Distanz} & \textbf{Abbildung} \\ \hline
    1 & (62.5, 61.0) & (62.0, 62.0) & 1.12 & \ref{fig:ball1} \\ \hline
    2 & (61.5, 61.5) & (61.0, 60.0) & 1.58 & \ref{fig:ball2} \\ \hline
    3 & (42.5, 43.0) & (41.0, 40.0) & 3.35 & \ref{fig:ball3} \\ \hline
    4 & (61.0, 61.0) & (60.0, 60.0) & 1.41 & \ref{fig:ball4} \\ \hline
    5 & (51.0, 48.5) & (51.0, 52.0) & 3.50 & \ref{fig:ball5} \\ \hline
    6 & (78.0, 76.0) & (78.0, 75.0) & 1.00 & \ref{fig:ball6} \\ \hline
    \end{tabular}
    \caption{Ergebnisse der Analyse der Schwerpunkte und Distanzen}
    \label{tab:centroid_distances}
\end{table}

\begin{minipage}[t]{0.3\textwidth}
    \includeImgNoUrl{H}{width=1.0\linewidth}{img/4_ball_1_centroid.png}{Schwerpunkte Ball 1}{fig:ball1}{\centering}
\end{minipage}
%
\begin{minipage}[t]{0.3\textwidth}
    \includeImgNoUrl{H}{width=1.0\linewidth}{img/4_ball_2_centroid.png}{Schwerpunkte Ball 2}{fig:ball2}{\centering}
\end{minipage}
%
\begin{minipage}[t]{0.3\textwidth}
    \includeImgNoUrl{H}{width=1.0\linewidth}{img/4_ball_3_centroid.png}{Schwerpunkte Ball 3}{fig:ball3}{\centering}
\end{minipage}


\begin{minipage}[t]{0.3\textwidth}
    \includeImgNoUrl{H}{width=1.0\linewidth}{img/4_ball_4_centroid.png}{Schwerpunkte Ball 4}{fig:ball4}{\centering}
\end{minipage}
%
\begin{minipage}[t]{0.3\textwidth}
    \includeImgNoUrl{H}{width=1.0\linewidth}{img/4_ball_5_centroid.png}{Schwerpunkte Ball 5}{fig:ball5}{\centering}
\end{minipage}
%
\begin{minipage}[t]{0.3\textwidth}
    \includeImgNoUrl{H}{width=1.0\linewidth}{img/4_ball_6_centroid.png}{Schwerpunkte Ball 6}{fig:ball6}{\centering}
\end{minipage}

