\section{Tests und Visualisierung}

Da die Mean Shift Berechnung sehr aufwendig ist, wurde für Testzwecke während der Entwicklung nur mit einem sehr kleinen 40x40 Bild und mit zweidimensionalen Daten aus einem CSV Datei getestet.


Folgendes Bild zeigt das Clustering der zweidimensionalen CSV-Daten. Aufgrund der farblichen Zuweisung der einzelnen Punkte zu den Clustern lässt sie hier bereits gut erkennen, dass das Clustering gut funktioniert.
\includeImgNoUrl{H}{width=0.7\textwidth}{img/mean_shift_result_csv_data.png}{Ergebnis mit Farbsegmentierung im 2D Bereich}{fig3}{\centering}


Für die Statusübersicht wird nach jedem Schritt der Fortschritt auf die Konsole ausgegeben.

% ```
% Start mean shift clustering with image: color_monkey_xs.jpg and bandwidth: 20.0 for 1600 pixels
% Iteration 1 (Max Iteration: 2000, Image size: 40 x 40):
%     Number of converged pixels: 11/1600
%     Average distance of the pixels from their centroid: 21.5346

% ...
% ```

Weiters wird pro Iterationsschritt, der aktuelle Zustand gespeichert, um diesen am Ende zu visualisieren und zu animieren.

Sämtliche Animationen sind als GIF im Anhang verfügbar.


Hier sind die Color Monkey Testbilder mit verschiedenen Bandwidth:

\begin{minipage}[t]{0.23\textwidth}
    \includeImgNoUrl{H}{width=0.95\linewidth}{img/color_monkey_xs_5.0_im.png}{Bandwidth 5.0}{fig1}{\centering}
\end{minipage}
%
\begin{minipage}[t]{0.23\textwidth}
    \includeImgNoUrl{H}{width=0.95\linewidth}{img/color_monkey_xs_10.0_im.png}{Bandwidth 10.0}{fig1}{\centering}
\end{minipage}
%
\begin{minipage}[t]{0.23\textwidth}
    \includeImgNoUrl{H}{width=0.95\linewidth}{img/color_monkey_xs_20.0_im.png}{Bandwidth 20.0}{fig1}{\centering}
\end{minipage}
%
\begin{minipage}[t]{0.23\textwidth}
    \includeImgNoUrl{H}{width=0.95\linewidth}{img/color_monkey_xs_30.0_im.png}{Bandwidth 30.0}{fig2}{\centering}
\end{minipage}



Hier sind die Bird Testbilder mit verschiedenen Bandwidth:

\begin{minipage}[t]{0.30\textwidth}
    \includeImgNoUrl{H}{width=0.95\linewidth}{img/bird_xs_5.0_im.png}{Bandwidth 5.0}{fig1}{\centering}
\end{minipage}
%
\begin{minipage}[t]{0.30\textwidth}
    \includeImgNoUrl{H}{width=0.95\linewidth}{img/bird_xs_10.0_im.png}{Bandwidth 10.0}{fig1}{\centering}
\end{minipage}
%
\begin{minipage}[t]{0.30\textwidth}
    \includeImgNoUrl{H}{width=0.95\linewidth}{img/bird_xs_40.0_im.png}{Bandwidth 40.0}{fig1}{\centering}
\end{minipage}


Testbilder vor und nach dem Mean Shift im Color Space:

Color Monkey:

\begin{minipage}[t]{0.45\linewidth}
    \includeImgNoUrl{H}{width=1.0\linewidth}{img/color_monkey_xs_30.0_cs_before.png}{Vor Mean Shift - Bandwidth 40.0}{fig1}{\centering}
\end{minipage}
%
\begin{minipage}[t]{0.45\linewidth}
    \includeImgNoUrl{H}{width=1.0\linewidth}{img/color_monkey_xs_30.0_cs_after.png}{Nach Mean Shift - Bandwidth 40.0}{fig1}{\centering}
\end{minipage}


Bird: 

\begin{minipage}[t]{0.45\linewidth}
    \includeImgNoUrl{H}{width=1.0\linewidth}{img/bird_xs_40.0_cs_before.png}{Vor Mean Shift - Bandwidth 40.0}{fig1}{\centering}
\end{minipage}
%
\begin{minipage}[t]{0.45\linewidth}
    \includeImgNoUrl{H}{width=1.0\linewidth}{img/bird_xs_40.0_cs_after.png}{Nach Mean Shift - Bandwidth 40.0}{fig1}{\centering}
\end{minipage}


Mit zu kleinen Bandwidth gibt es kein gutes Ergebnis:

\begin{minipage}[t]{0.45\linewidth}
    \includeImgNoUrl{H}{width=1.0\linewidth}{img/color_monkey_xs_10.0_cs_after.png}{Nach Mean Shift - Bandwidth 10.0}{fig1}{\centering}
\end{minipage}
%
\begin{minipage}[t]{0.45\linewidth}
    \includeImgNoUrl{H}{width=1.0\linewidth}{img/bird_xs_10.0_cs_after.png}{Nach Mean Shift - Bandwidth 10.0}{fig1}{\centering}
\end{minipage}



\newpage

