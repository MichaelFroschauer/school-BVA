\section{Teilüberschrift}


\begin{table}[!htp]
  \centering
  \begin{tabular}{|c|c|c|c|c|c|c|}
  \hline
  \textbf{Nr} & \textbf{L} & \textbf{I} & \textbf{T} & \textbf{Summe} & \textbf{Multiplikator} & \textbf{Punkte} \\ \hline
  1a          & 3/3        & 4/4        & 3/3        & 10             & 3                      & 30              \\ \hline
  1b          & 3/3        & 4/4        & 3/3        & 10             & 3                      & 30              \\ \hline
  2a          & 3/3        & 4/4        & 3/3        & 10             & 1                      & 10              \\ \hline
  2b          & 3/3        & 4/4        & 3/3        & 10             & 2                      & 20              \\ \hline
  2c          & 3/3        & 4/4        & 3/3        & 10             & 1                      & 10              \\ \hline
  \end{tabular}
\end{table}

\subsection{Aufzählungen}
\begin{itemize}
  \item TESTITEM
\end{itemize}

%include code
%\lstinputlisting[caption=Scheduler, style=customc]{sudoku_solver.c}
% or
%\UseRawInputEncoding
%\lstinputlisting[language=SQL, firstline=1, lastline=100]{../src/2_SQL-Wiederholung.sql}
% or
%\lstinputlisting[language=c, firstline=1, lastline=100]{sudoku_solver.c}

% \begin{minipage}[t]{0.4\textwidth}
%   \includeImgNoUrl{H}{width=0.4\linewidth}{img/3_ball_maske_farbe.png}{Ballmaske mit Farbsegmentierung}{fig1}{\centering}
% \end{minipage}
% %
% \begin{minipage}[t]{0.6\textwidth}
%   \includeImgNoUrl{H}{width=0.8\linewidth}{img/3_ball_maske_farbe_1.png}{Ball Overlay mit Farbsegmentierung}{fig2}{\centering}
% \end{minipage}


%include code with minted
%\begin{minted}[linenos]{cpp}
%    for i in range(0,2):
%        print(i)
%\end{minted}

%\inputminted[]{java}{../src/src/at/fhooe/swe4/queues/Heap.java}

%minted with caption
%\begin{listing}[!ht]
%\caption{summary\_spezial.xml - Erstellt aus ue4\_fitnessdokument\_spezial.xml}
%\inputminted[]{xml}{../src/XMLWizard/xmlFiles/summary_spezial.xml}
%\end{listing}


%include svg
%\begin{figure}[!htbp]
%  \centering
%  \includesvg[inkscapelatex=false]{../svg/1.1_Bestellung.svg}
%  \caption{UML-Bestellungen}
%\end{figure}

%include picture
%\includegraphics[]{../img/2.3.png} 
%\includegraphics[width=0.9\textwidth]{../img/5.grouping_sets.png}
%\includegraphics[scale=0.5]{./img/mock/chatSearch.png} 

% \begin{figure}[!htbp]
%   \centering
%   \includegraphics[scale=0.5]{./img/mock/chatSearch.png} 
%   \caption{Login Window}
% \end{figure}

% \begin{minipage}[t]{0.65\textwidth}
%   \vspace{0.5cm}
%   Für 2 beliebige Algorithmen 1,2 gibt es immer Probleme A,B, bei denen: \\
%   - Algorithmus 1 bei Problem 1 besser abschneidet als Algorithmus 2 \\
%   - Algorithmus 2 bei Problem 2 besser abschneidet als Algorithmus 1.
% \end{minipage}
% %
% \begin{minipage}[t]{0.35\textwidth}
%   \includeImgNoUrl{H}{width=0.8\linewidth}{img/2_no_free_lunch.png}{No Free Lunch Theorem}{fig:2_no_free_lunch}{\centering}
% \end{minipage}

% Formeln in Latex
%  \begin{align*}
%  Kosten &= FK + VK*A_t \\
%  Ertrag &= A_t*\frac{B_k}{A} \\
%  Gewinn &= Ertrag - Kosten \\
%  VK &= WK / SW \\
%  \frac{B_k}{A} &= \frac{B}{A}+ \frac{B}{A}*(\frac{A}{min} - \frac{A_t}{min})
%  %B_k/A &= B/A + B/A * (A/min - A_t/min)
%  \end{align*}


% Eine einfache Grafik
% \begin{figure}[h]
%   \centering
%   \includegraphics[scale=0.7]{Business_model_innovation}
%   \caption[Who-what-how-why-Diagramm]{ \textbf{Who-what-how-why}-Diagramm aus dem St. Gallen Business Modell Navigator \cite{gassmann2020geschäftsmodelle}}
% \end{figure}


% Zwei Grafiken nebeneinander
%  \begin{figure}[H]
%    \centering
%    \begin{subfigure}{.5\textwidth}
%      \centering
%      \includegraphics[clip, trim=0cm 4.5cm 0cm 2.1cm, scale=0.29]{view2}
%      \caption{Ansicht im dreidimensionalen Raum}
%      \label{fig:sub1}
%    \end{subfigure}%
%    \begin{subfigure}{.5\textwidth}
%      \centering
%      \includegraphics[clip, trim=0cm 4.5cm 0cm 2.1cm, scale=0.29]{view5}
%      \caption{Gegenüberstellung: Anzahl der Ausführungen $\leftrightarrow$ Gewinn}
%      \label{fig:sub2}
%    \end{subfigure}
%    \caption[Gewinnfunktion]{Zwei Ansichten der Gewinnfunktion. Eine Steigerung der Anzahl der Ausführungen bzw. eine Erhöhung des Betrags/Ausführung wirkt sich positiv auf den Gewinn aus. Das Optimum im zitronengelben Bereich des Funktionsgraphen ist nur in der Modellrechnung erreichbar. $\rightarrow$ die Kombination \textit{hoher Betrag} * \textit{hohe Anzahl an Ausführungen} ist für keinen Kunden finanziell attraktiv.}
%    \label{Gewinnfunktion}
%  \end{figure}

\newpage
