\section{Implementierung}
Im Allgemeinen war bereits ein Großteil des benötigten Quellcodes mit der Angabe gegeben. Der gegebene Quelltext wurde um einige Methoden und Funktionen erweitert
um die benötigte Funktion abbilden zu können. Bei der Implementierung wurde laut Angabe vorgegangen. \\

Zusätzlich wird der Code um automatisierte Tests sowie Funktionen zur visuellen Kontrolle ergänzt. Zur Überprüfung der Erkennung kann ein Overlay erzeugt werden, das die detektierten Zeichenbereiche im Originalbild hervorhebt. Debugging-Funktionen sind implementiert, jedoch im finalen Code standardmäßig auskommentiert.

\subsection{Zerteilen des Bildes in Teilbilder je Zeichen}
Das Eingabebild wird zunächst in Graustufen eingelesen und binarisiert. Anschließend erfolgt eine zweistufige Segmentierung der Zeichen:

\begin{enumerate}
  \item \textbf{Horizontale Segmentierung:} Es werden Bildzeilen identifiziert, in denen Zeichen vorhanden sind. Dabei wird geprüft, ob eine Bildreihe ausschließlich Hintergrund enthält.
  \item \textbf{Vertikale Segmentierung:} In den gefundenen Zeilen werden die einzelnen Zeichen durch Analyse der Spalten separiert. Auch hier wird überprüft, ob eine Spalte nur Hintergrund enthält.
\end{enumerate}

Für jedes segmentierte Zeichen wird ein \texttt{SubImageRegion}-Objekt erzeugt, welches die Position, Höhe und Breite des Bereichs im Originalbild beschreibt. Zusätzlich wird eine Methode zur minimalen Höhenanpassung (\texttt{minimize\_character\_bounding\_height}) angewendet, um den Zeichenbereich vertikal anzupassen.

\subsection{Klassifizierung}
Um die Teilbilder der einzelnen Zeichen einem konkreten Zeichen zuzuordnen, werden Features verwendet. Diese Features sind Merkmale, anhand welcher Zeichen 
identifiziert werden können. Die Kombination dieser Features soll in Folge möglichst alle Zeichen eindeutig klassifizieren. Um alle Features im Ergebnis
gleich zu gewichten, müssen diese normalisiert werden. Diese wurde bereits in der Angabe mitgeliefert. Folgende Features sind implementiert:
\begin{itemize}
    \item \textbf{FGcount}: Zählt die Anzahl der Vordergrundpixel (also die Zeichenpixel) innerhalb eines Zeichenbereichs.
    \item \textbf{MaxDistX}: Absolute Breite der Bounding-Box
    \item \textbf{MaxDistY}: Absolute Höhe der Bounding-Box
    \item \textbf{AspectRatio}: Relatives Verhältnis zwischen Breite und Höhe
    \item \textbf{FgBgRatio}: Relatives Verhältnis zwischen Vordergrund und Hintergrund
    \item \textbf{VerticalAsym}: Symetrie auf der vertikalen Achse (Es wird eine lineare Funktion vom linken unteren Pixel zum rechten oberen erzeugt $\Rightarrow f(x) = kx + d$, wobei d immer 0 ist, und jeder Pixel mit diesem Gewichtsfaktor aufsummiert)
    \item \textbf{HorizontalAsym}: Symetrie auf der horizontalen Achse (Es wird eine lineare Funktion vom linken oberen Pixel zum rechten unteren erzeugt $\Rightarrow f(x) = kx + d$, wobei d immer 0 ist, und jeder Pixel mit diesem Gewichtsfaktor aufsummiert)
\end{itemize}
