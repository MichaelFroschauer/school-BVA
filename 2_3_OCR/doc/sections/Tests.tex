\section{Test}
\label{sec-tests}
Um tests einfach wiederholbar zu gestalten sind diese in Form von statisch definierten Testfällen implementiert. Inhalt dieser Testfälle sind sämtliche Vorkommnisse 
aller Zeichen im Bespiel-Bild. Die Testfälle prüfen dabei immer, ob das erwartete Vorkommen eines Zeichens auch tatsächlich gefunden wurde. Um eine solche Testmöglichkeit
bereitzustellen ist die gegebene "run"-Methode entsprechend erweitert:
\begin{itemize}
    \item Es wird die Anzahl an gefunden Zeichen als Ergebnis zurückgegeben
    \item Die Schnittstelle bietet die Möglichkeit das Referenzzeichen von extern parameterieren zum können
\end{itemize}

\begin{lstlisting}{language=Python}
def run(self, img_path, tgtCharRow: int, tgtCharCol: int) -> int:
    ...
\end{lstlisting}

Die Testfälle an sich können für das gegebene Testbild direkt via der main Methode ausgeführt werden. Zu diesen \glqq automatisierten\grqq\space Tests 
wurden zusätzlich noch manuelle Tests durchgeführt um das Splitten, Minimieren und Einfärben der Bounding-Boxes zu testen. Diese Tests sind manuell 
durchzuführen und optisch zu prüfen.

\subsection{Ergebnis}
\subsubsection{Automatisierte Tests}
\begin{lstlisting}{}
    SUCCESS character "e" - expected: 169 found: 169
    SUCCESS character "n" - expected: 115 found: 115
    SUCCESS character "s" - expected: 102 found: 102
    SUCCESS character "a" - expected: 92 found: 92
    FAILED character "t" - expected: 82 found: 78
    FAILED character "r" - expected: 81 found: 75
    SUCCESS character "d" - expected: 69 found: 69
    SUCCESS character "i" - expected: 57 found: 57
    SUCCESS character "h" - expected: 45 found: 45
    SUCCESS character "u" - expected: 39 found: 39
    SUCCESS character "o" - expected: 38 found: 38
    SUCCESS character "m" - expected: 37 found: 37
    SUCCESS character "l" - expected: 35 found: 35
    SUCCESS character "c" - expected: 33 found: 33
    SUCCESS character "g" - expected: 27 found: 27
    FAILED character "w" - expected: 25 found: 24
    SUCCESS character "G" - expected: 23 found: 23
    SUCCESS character "b" - expected: 22 found: 22
    SUCCESS character "." - expected: 20 found: 20
    SUCCESS character "," - expected: 16 found: 16
    SUCCESS character "L" - expected: 10 found: 10
    FAILED character "v" - expected: 10 found: 9
    SUCCESS character "E" - expected: 8 found: 8
    SUCCESS character "W" - expected: 8 found: 8
    SUCCESS character ":" - expected: 8 found: 8
    SUCCESS character "T" - expected: 8 found: 8
    SUCCESS character "D" - expected: 8 found: 8
    SUCCESS character "S" - expected: 8 found: 8
    SUCCESS character "A" - expected: 7 found: 7
    SUCCESS character "ü" - expected: 7 found: 7
    SUCCESS character "ö" - expected: 7 found: 7
    FAILED character "f" - expected: 6 found: 5
    SUCCESS character "F" - expected: 6 found: 6
    FAILED character "z" - expected: 6 found: 5
    SUCCESS character "H" - expected: 5 found: 5
    SUCCESS character "p" - expected: 5 found: 5
    SUCCESS character "M" - expected: 4 found: 4
    SUCCESS character "B" - expected: 3 found: 3
    SUCCESS character ";" - expected: 2 found: 2
    SUCCESS character "U" - expected: 2 found: 2
    SUCCESS character "N" - expected: 2 found: 2
    SUCCESS character "k" - expected: 2 found: 2
    SUCCESS character "j" - expected: 2 found: 2
    SUCCESS character "P" - expected: 2 found: 2
    SUCCESS character "ä" - expected: 2 found: 2
    SUCCESS character "I" - expected: 1 found: 1
    SUCCESS character "O" - expected: 1 found: 1
    SUCCESS character "Z" - expected: 1 found: 1
    SUCCESS character "J" - expected: 1 found: 1
    ==============================================================
    Tests passsed: SUCCESSFULLY 43 / FAILED 6
    ==============================================================
\end{lstlisting}

Die Tests mit dem gesamten Zeichensatz des Testbilds zeigten eine hohe Erkennungsgenauigkeit. Insgesamt wurden 49 Zeichenklassen getestet, wobei 43 erfolgreich und 6 fehlerhaft erkannt wurden. Diese Problemfälle sind zurückzuführen auf die \glqq primitive\grqq\space splitting Methode(Fire-Through-Methode), welche für das Beispiel-Bild nicht immer alle Zeichen richtig zerteilt. Es gibt Zeichenfolgen,
die nicht zum Teil überlappend sind und somit nicht richig getrennt werden. Diese Fehler werden in der folgenden Tabelle genauer gezeigt.
\begin{table}[H]
    \begin{tabularx}{\linewidth}{| l | X |}
        \hline
        rt & \includeImgNoUrl{H}{width=0.2\textwidth}{img/r-t-fehler.png}{Fehlerhafte Segmentierung: \textit{r} und \textit{t} werden als ein Zeichen erkannt}{fig-fehler1}{\centering}\\\hline
        rv & \includeImgNoUrl{H}{width=0.25\textwidth}{img/r-v-fehler.png}{Fehlerhafte Segmentierung: \textit{r} und \textit{v}}{fig-fehler2}{\centering}\\\hline
        rf & \includeImgNoUrl{H}{width=0.2\textwidth}{img/r-f-fehler.png}{Fehlerhafte Segmentierung: \textit{r} und \textit{f}}{fig-fehler3}{\centering}\\\hline
        zw & \includeImgNoUrl{H}{width=0.3\textwidth}{img/z-w-fehler.png}{Fehlerhafte Segmentierung: \textit{z} und \textit{w} verschmelzen zu einem Zeichen}{fig-fehler4}{\centering}\\\hline
    \end{tabularx}
\end{table}

\textbox{
    In Summe sind davon alle Zeichen betroffen, welche in der Auswertung fehlgeschlagen sind.
}

\subsubsection{Beispiel: Test von erkannten e}
In diesem Beispiel ist demonstriert wie beispielsweise das Zeichen \glqq e \grqq\space erkannt wird. Die identifizierten Zeichen werden hier in Graustufe hinterlegt.
\includeImgNoUrl{H}{width=\linewidth}{img/e_ok.png}{}{}{\centering}

\subsubsection{Beispiel: Test von Bounding-Box Minimierung}
In diesem Beipsiel wird demonstrativ überprüft, ob das minimieren der Bounding-Boxen entsprechend funktioniert. Als Beispielwort ist mit Absicht das Wort Früchte
ausgewählt, da dieses einige Besonderheiten wie z.B.: das Zeichen ü aufweißt.
\includeImgNoUrl{H}{width=0.6\linewidth}{img/bbox_beispiel.png}{}{}{\centering}


