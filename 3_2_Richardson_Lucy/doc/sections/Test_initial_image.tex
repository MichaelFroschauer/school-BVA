\section{Test mit verschiedenen Startbildern und Filterkernels}

In diesem Kapitel werden die Ergebnisse der Richardson-Lucy-Dekonvolution mit 100 Iterationen unter verschiedenen Bedingungen dargestellt. Ziel ist es, die Auswirkungen der gewählten \textit{Point Spread Function, PSF} sowie des verwendeten Startbildes (Initialisierung) auf die Qualität der restaurierten Bilder zu untersuchen.

\begin{itemize}
    \item \textbf{Random Guess Estimation}: Ein zufällig erzeugtes Bild. Das Ergebnis zeigt starke Artefakte und Konvergenzprobleme.
    \item \textbf{Gray Guess Estimation}: Ein konstant graues Bild als Startwert. Das Ergebnis ist bereits relativ gut.
    \item \textbf{Blurred Image Guess Estimation}: Das verschwommene Bild selbst wurde als Initialisierung verwendet, es ist sehr ähnlich zum grauen Startbild.
    \item \textbf{Duck Image Guess Estimation}: Ein Bild von Donald Duck wurde zur Initialisierung verwendet.
\end{itemize}


\subsection{Mittelwertfilter als PSF}

\subsubsection{Originalbilder}

In Abbildung \ref{fig:original_1} und \ref{fig:mean_blur_1} sind die Ausgangsbilder dargestellt, welche als Basis für diesen Test dienten. Das erste Bild stellt das Originalbild dar. Dieses wurde anschließend mit einem Mittelwertfilter gefaltet, um eine Unschärfe zu erzeugen.

\noindent
\begin{minipage}[t]{0.5\textwidth}
    \includeImgNoUrl{H}{width=0.7\linewidth}{img/original.png}{Originalbild}{fig:original_1}{\centering}
\end{minipage}
%
\begin{minipage}[t]{0.5\textwidth}
    \includeImgNoUrl{H}{width=0.7\linewidth}{img/mean_blurred.png}{Mit Mittelwertfilter (PSF) gefiltert}{fig:mean_blur_1}{\centering}
\end{minipage}

\subsubsection{Restaurierte Bilder ohne Rauschen}

Die Abbildungen \ref{fig:mean_rand} bis \ref{fig:mean_duck} zeigen die Ergebnisse der Dekonvolution ohne Rauschen, wobei der Mittelwertfilter als PSF verwendet wurde.

\paragraph{Bewertung:} 
Die Ergebnisse zeigen, dass die Wahl des Startbildes einen Einfluss auf die Qualität der Rekonstruktion hat. Die \textit{Random Guess}-Initialisierung führt zu deutlich sichtbaren Artefakten im rekonstruierten Bild. Dagegen liefern sowohl die \textit{Gray Guess}- als auch die \textit{Blurred Guess}-Variante deutlich bessere Resultate mit wenigen Artefakten. Bei der \textit{Duck Guess}-Initialisierung ist noch ein Teil des Donald-Duck-Bildes in der Rekonstruktion erkennbar, was die Qualität beeinträchtigt. \\

\noindent
Die Initialisierung erfolgte mit verschiedenen Startbildern:

\noindent
\begin{minipage}[t]{0.5\textwidth}
    \includeImgNoUrl{H}{width=0.7\linewidth}{img/restored_mean_random.png}{Random Guess Estimation}{fig:mean_rand}{\centering}
\end{minipage}
%
\begin{minipage}[t]{0.5\textwidth}
    \includeImgNoUrl{H}{width=0.7\linewidth}{img/restored_mean_gray.png}{Gray Guess Estimation}{fig:mean_gray}{\centering}
\end{minipage}

\noindent
\begin{minipage}[t]{0.5\textwidth}
    \includeImgNoUrl{H}{width=0.7\linewidth}{img/restored_mean_blurred.png}{Blurred Image Guess Estimation}{fig:mean_blurred}{\centering}
\end{minipage}
%
\begin{minipage}[t]{0.5\textwidth}
    \includeImgNoUrl{H}{width=0.7\linewidth}{img/restored_mean_donaldDuck.png}{Duck Image Guess Estimation}{fig:mean_duck}{\centering}
\end{minipage}


\newpage
\subsection{Gauß-Filter als PSF}

\subsubsection{Originalbilder}

In Abbildung \ref{fig:original_2} und \ref{fig:gauss_blurred_1} sind die Ausgangsbilder dargestellt, welche als Basis für diesen Test dienten. Das erste Bild stellt das Originalbild dar. Dieses wurde anschließend mit einem Gauß-Filter gefaltet, um eine Unschärfe zu erzeugen.

\noindent
\begin{minipage}[t]{0.5\textwidth}
    \includeImgNoUrl{H}{width=0.7\linewidth}{img/original.png}{Originalbild}{fig:original_2}{\centering}
\end{minipage}
%
\begin{minipage}[t]{0.5\textwidth}
    \includeImgNoUrl{H}{width=0.7\linewidth}{img/gauss_blurred.png}{Mit Gaussfilter (PSF) gefiltert}{fig:gauss_blurred_1}{\centering}
\end{minipage}

\subsubsection{Restaurierte Bilder ohne Rauschen}

Bei Verwendung eines Gauß-Kernels als PSF zeigen sich (Abbildungen \ref{fig:gauss_rand} bis \ref{fig:gauss_duck}) ähnliche Tendenzen wie beim Mittelwertfilter. 

\paragraph{Bewertung:}
Auch beim Gauß-Filter hängt die Qualität der Rekonstruktion stark von der Initialisierung ab. Die \textit{Random Guess}-Variante zeigt erneut viele Artefakte. \textit{Gray Guess} und \textit{Blurred Guess} liefern brauchbare Ergebnisse, wirken jedoch etwas verschwommener als beim Mittelwertfilter. Bei der \textit{Duck Guess}-Initialisierung bleibt das ursprüngliche Donald-Duck-Motiv deutlich sichtbar, was zu einer fehlerhaften Rekonstruktion führt.

\noindent
\begin{minipage}[t]{0.5\textwidth}
    \includeImgNoUrl{H}{width=0.7\linewidth}{img/restored_gauss_random.png}{Random Guess Estimation}{fig:gauss_rand}{\centering}
\end{minipage}
%
\begin{minipage}[t]{0.5\textwidth}
    \includeImgNoUrl{H}{width=0.7\linewidth}{img/restored_gauss_gray.png}{Gray Guess Estimation}{fig:gauss_gray}{\centering}
\end{minipage}

\noindent
\begin{minipage}[t]{0.5\textwidth}
    \includeImgNoUrl{H}{width=0.7\linewidth}{img/restored_gauss_blurred.png}{Blurred Image Guess Estimation}{fig:gauss_blurred}{\centering}
\end{minipage}
%
\begin{minipage}[t]{0.5\textwidth}
    \includeImgNoUrl{H}{width=0.7\linewidth}{img/restored_gauss_donaldDuck.png}{Duck Image Guess Estimation}{fig:gauss_duck}{\centering}
\end{minipage}




\newpage
\subsection{Horizontal-Filter als PSF}

\subsubsection{Originalbilder}

Die Originalbilder wurden mithilfe eines horizontalen PSFs gefiltert. Dabei wird vor allem horizontale Information verschmiert, wodurch vertikale Kanten stärker betroffen sind.

\noindent
\begin{minipage}[t]{0.5\textwidth}
    \includeImgNoUrl{H}{width=0.7\linewidth}{img/original.png}{Originalbild}{fig:original}{\centering}
\end{minipage}
%
\begin{minipage}[t]{0.5\textwidth}
    \includeImgNoUrl{H}{width=0.7\linewidth}{img/horizontal_blurred.png}{Mit horizontalem Filter (PSF) gefiltert}{fig:mean_blur}{\centering}
\end{minipage}

\subsubsection{Restaurierte Bilder ohne Rauschen}

\paragraph{Bewertung:}
Die Ergebnisse zeigen, dass der horizontale PSF erwartungsgemäß zu vertikalen Artefakten in der Rekonstruktion führt. Die \textit{Random Guess}-Initialisierung resultiert erneut in vielen Störungen, wobei ein vertikales Muster erkennbar ist. \textit{Gray Guess} und \textit{Blurred Guess} liefern vergleichbare, aber durch vertikale Streifen beeinträchtigte Resultate. Die \textit{Duck Guess}-Initialisierung führt zu einem besonders unscharfen Ergebnis, bei dem die Struktur des Entenbildes noch erkennbar, aber deutlich verschmiert ist.

\noindent
\begin{minipage}[t]{0.5\textwidth}
    \includeImgNoUrl{H}{width=0.7\linewidth}{img/restored_horizontal_random.png}{Random Guess Estimation}{fig:horizontal_rand}{\centering}
\end{minipage}
%
\begin{minipage}[t]{0.5\textwidth}
    \includeImgNoUrl{H}{width=0.7\linewidth}{img/restored_horizontal_gray.png}{Gray Guess Estimation}{fig:horizontal_gray}{\centering}
\end{minipage}

\noindent
\begin{minipage}[t]{0.5\textwidth}
    \includeImgNoUrl{H}{width=0.7\linewidth}{img/restored_horizontal_blurred.png}{Blurred Image Guess Estimation}{fig:horizontal_blurred}{\centering}
\end{minipage}
%
\begin{minipage}[t]{0.5\textwidth}
    \includeImgNoUrl{H}{width=0.7\linewidth}{img/restored_horizontal_donaldDuck.png}{Duck Image Guess Estimation}{fig:horizontal_duck}{\centering}
\end{minipage}




\subsection{Diagonal-Filter als PSF}

\subsubsection{Originalbilder}

Durch Anwendung eines diagonalen PSFs wird Bildinformation entlang einer Diagonale verschmiert, was zu einer Richtungsunschärfe führt.

\noindent
\begin{minipage}[t]{0.5\textwidth}
    \includeImgNoUrl{H}{width=0.7\linewidth}{img/original.png}{Originalbild}{fig:original}{\centering}
\end{minipage}
%
\begin{minipage}[t]{0.5\textwidth}
    \includeImgNoUrl{H}{width=0.7\linewidth}{img/diagonal_blurred.png}{Mit Mittelwertfilter (PSF) gefiltert}{fig:mean_blur}{\centering}
\end{minipage}


\subsubsection{Restaurierte Bilder ohne Rauschen}

\paragraph{Bewertung:}
Die Ergebnisse ähneln jenen des horizontalen PSFs, jedoch treten die Artefakte entsprechend der diagonalen Verschmierungsrichtung auf. Bei \textit{Random Guess} zeigen sich deutliche diagonale Muster und Artefakte. \textit{Gray Guess} und \textit{Blurred Guess} liefern insgesamt stabile, aber leicht diagonale Streifen aufweisende Resultate. Die \textit{Duck Guess}-Initialisierung führt erneut zu einem stark verschmierten Bild, bei dem die ursprüngliche Struktur nur noch schwach zu erkennen ist.

\noindent
\begin{minipage}[t]{0.5\textwidth}
    \includeImgNoUrl{H}{width=0.7\linewidth}{img/restored_diagonal_random.png}{Random Guess Estimation}{fig:diagonal_rand}{\centering}
\end{minipage}
%
\begin{minipage}[t]{0.5\textwidth}
    \includeImgNoUrl{H}{width=0.7\linewidth}{img/restored_diagonal_gray.png}{Gray Guess Estimation}{fig:diagonal_gray}{\centering}
\end{minipage}

\noindent
\begin{minipage}[t]{0.5\textwidth}
    \includeImgNoUrl{H}{width=0.7\linewidth}{img/restored_diagonal_blurred.png}{Blurred Image Guess Estimation}{fig:diagonal_blurred}{\centering}
\end{minipage}
%
\begin{minipage}[t]{0.5\textwidth}
    \includeImgNoUrl{H}{width=0.7\linewidth}{img/restored_diagonal_donaldDuck.png}{Duck Image Guess Estimation}{fig:diagonal_duck}{\centering}
\end{minipage}